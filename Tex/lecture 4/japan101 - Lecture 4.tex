\documentclass{article}

\usepackage[utf8]{inputenc}
\usepackage{fullpage}
\usepackage{times}
\usepackage{tcolorbox}
\usepackage{enumitem}
\usepackage{bookmark}
\usepackage{CJKutf8}

\setlist{itemsep=2pt}
\renewcommand{\thesection}{Lecture \arabic{section}}
\renewcommand{\thesubsection}{\arabic{section}.\arabic{subsection}}
\setcounter{section}{3}

\begin{document}
 \begin{CJK}{UTF8}{min}

\noindent
{JAPAN 101 \hfill Hao Pan}\\
{Shimoda, Fumie}\\
{Spring 2018}

%%%%%%%%%%%%%%%%%%%%%%%%%%%%%%%%%%%

\begin{center}
\section{}
\noindent
{\hfill 23/05/2018 [W]}
\end{center}

\subsection{Places}

The following are nouns used to address locations:

\begin{tabular}{ l | l | l }
Japanese & English & Situation\\
\hline
Koko & Here & You are close\\
Soko & There & They are close\\
Asoko & There & Neither you nor them are close\\
Doko & Where & -
\end{tabular}

\begin{itemize}
\item {\bf Ex.} {\it Sumimasen. [N] wa doko desu ka. | Excuse me, where is [N]?}
\end{itemize}

\subsection{Floors (of a building)}

\begin{tabular}{ c | l }
1 & ik-kai\\
2 & ni-kai\\
3 & san-gai\\
4 & yon-kai
\end{tabular}\\

\subsection{New Vocabulary}

\begin{itemize}
\item {\bf [item]-ya} means a store that sells [item]. {\bf ex.} {\it booshi-ya | hat store}.
\item も means "too" or "also".
\begin{itemize}
\item {\bf Ex.} まえりさんはがくせいです。\\
		たけしさんもがくせいです。
\end{itemize}
\end{itemize}

\subsection{Negative Sentence}

\begin{itemize}
\item じゃないです
\item {\bf Ex.}
\begin{itemize}
\item たけしさんはにほんじんです。
\item まえりさんはにほんじんじゃないです。
\end{itemize}
\item じゃない can also be used in a negative statement.
\begin{itemize}
\item いいえ。そうじゃないです。
\end{itemize}
\end{itemize}

\subsection{Tags}

\begin{tabular}{ l | l | l }
character & usage & meaning\\
ね& add to end of sentence & isn't it?\\
よ & add to end of sentence & I tell you.\\
\end{tabular}\\

\bigskip

\underline{\bf Examples}
\begin{itemize}
\item このひとは Justin Bieber ですね。
\item このひとはさくらいですよ。
\end{itemize}






\end{CJK}
\end{document}