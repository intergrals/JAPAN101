\documentclass{article}

\usepackage[utf8]{inputenc}
\usepackage{fullpage}
\usepackage{times}
\usepackage{tcolorbox}
\usepackage{enumitem}
\usepackage{multicol}
\usepackage{bookmark}
\usepackage{CJKutf8}
\usepackage[normalem]{ulem}

\bookmarksetup{
  numbered, 
  open,
}

\setlist{itemsep=2pt}
\renewcommand{\thesection}{Lecture \arabic{section}}
\renewcommand{\thesubsection}{\arabic{section}.\arabic{subsection}}
\setcounter{section}{6}

\begin{document}
 \begin{CJK}{UTF8}{min}

\noindent
{JAPAN 101 \hfill Hao Pan}\\
{Shimoda, Fumie}\\
{Spring 2018}

%%%%%%%%%%%%%%%%%%%%%%%%%%%%%%%%%%%

\begin{center}
\section{}
\noindent
{\hfill 13/06/2018 [W]}
\end{center}

\subsection{Days, Weeks, Months}

\begin{multicols}{3}

\subsubsection{Months}
\begin{tabular}{ | c | l | }
\hline
Month \# & Japanese\\
\hline
\# & [\#] がつ\\
4 & しがつ\\
7 & しちがつ\\
9 & くがつ\\
\hline
\end{tabular}

\vfill\null
\columnbreak

\subsubsection{Day}
\begin{tabular}{ | c | l | }
\hline
Day \# & Japanese\\
\hline
\# & [\#] にち\\
1 & ついたち\\
2 & ふつか\\
3 & みっか\\
4 & よっか\\
5 & いつか\\
6 & むいか\\
7 & なのか\\
8 & ようか\\
9 & ここのか\\
10 & とおか\\
14 & じゅうよっか\\
20 & はつか\\
24 & にじゅうよっか\\
\hline
\end{tabular}

\vfill\null
\columnbreak

\subsubsection{Weekday}

$\left.
\begin{tabular}{ | c | l | }
\hline
Day & Japanese\\
\hline
S & にち\\
M & げつ\\
T & か\\
W & すい\\
Th & もく\\
F & きん\\
Sa & ど\\
\hline
\end{tabular}
\right\}$ + ようび

\end{multicols}

\begin{itemize}
\item {\bf Ex.} きょうは\uline{ろく}がつ\uline{じゅうさん}(にち)です。 | {\it Today is \uline{June} \uline{13th}}.
\item \uline{たんじょうび}は\uline{く}がつ\uline{じゅうご}にちです。 | {\it My \uline{birthday} is on \uline{September} \uline{15th}.}
\item {\bf Q}: たんじょうびは\uline{いつ}ですか。 | {\it \uline{When} is your birthday?}
\begin{itemize}
\item {\bf A}: __がつ__(にち)です。
\end{itemize}
\end{itemize}

%%%%%%%%%%%%%%%%%%%%%%%%%%%%%%%%%%%
\subsection{Existence of Things/People}

[N] が
$\left\{
\begin{tabular}{@{} l l @{}}
あります。 & (Inanimate objects)\\
います。 & (Animate objects)\\
\end{tabular}
\right.$
\begin{itemize}
\item {\bf Translation}: There is [N].
\item {\bf Examples}:
\begin{itemize}
\item つくえがあります。 | {\it There is a desk.}
\item いぬがいません。 | {\it There is no dog.}
\end{itemize}
\item For items like fish (さかな), sentence depends on context (whether fish is alive or dead).
\item Trees and flowers are considered inanimate.
\end{itemize}

\clearpage

\subsubsection{Add Location}

[Place] に…

\begin{itemize}
\item {\bf Ex.} \uline{このへや}につくえがあります。 | {\it \uline{This room} has desks.}
\item {\bf Q}: \uline{あなた}のへや\uline{てれび}がありますか。 | {\it Is there a \uline{TV} in \uline{your room}?}
\begin{itemize}
\item {\bf A}: はい、あります。 | {\it Yes there is.}
\item {\bf A}: いいえ、ありません。 | {\it No there isn't.}
\end{itemize}
\item {\bf Q}: あなたのへやに\uline{なに}がありますか。 | \emph{\uline{What} is there in your room?}
\begin{itemize}
\item {\bf A}: [N] があります。 | \emph{There is \emph{[N]}.}
\end{itemize}
\item {\bf Q}: あなたのうちにだれがいますか。 | {\it Who is in your house?}
\begin{itemize}
\item {\bf A}: [N] がいます。 | \emph{\emph{[N]} is there.}
\end{itemize}
\end{itemize}

%%%%%%%%%%%%%%%%%%%%%%%%%%%%%%%%%%%
\subsection{Time and Events}

[T] に [E] があります。

\begin{itemize}
\item \uline{すいようび}に\uline{くらす}があります。 | {\it I have \uline{class} on \uline{Wednesday}.}
\item {\bf Q}: \uline{すいようび}に\uline{なに}がありますか。 | {\it \uline{What} do you have on \uline{Wednesday}?}
\begin{itemize}
\item {\bf A}: \uline{ふらんすご}と\uline{えいご}と\uline{こんぴゅうたあ}の\uline{くらす}があります。 | {\it I have \uline{French}, \uline{English}, and \uline{computer} \uline{classes}}.
\end{itemize}
\end{itemize}

%%%%%%%%%%%%%%%%%%%%%%%%%%%%%%%%%%%
\subsection{Describing Where Things Are}

\begin{itemize}
\item {\bf Q}: [N1] はどこですか。 | \emph{Where is \emph{[N1]}?}
\begin{itemize}
\item {[N2]} の [Location] です。 | \emph{It is \emph{[Location] [N2]}.} (location relative to N2).
\end{itemize}
\end{itemize}

\begin{tabular}{ | l | l || l | l | }
\hline
\multicolumn{4}{ | c | }{Location Words}\\
\hline
うえ & On top & なか & Inside\\
した & Under & そと & Outside\\
みぎ & Right of & となり & Next to\\
ひたり & Left of & ちかる & Close to\\
まえ & In front of & あいだ & Between\\
うしろ & Behind & - & -\\
\hline
\end{tabular}

\begin{itemize}
\item {\bf Q}: とけいは\uline{どき}ですか。 | {\it \uline{Where} is the clock.}
\begin{itemize}
\item {\bf A}: てれびの\uline{まえ}です。 | {\it \uline{In front} of the TV.}
\end{itemize}
\item {[N2]} と [N3] の\uline{あいだ}です。 | \emph{It is \uline{between} \emph{[N2]} and \emph{[N3]}.}
\end{itemize}




\end{CJK}
\end{document}