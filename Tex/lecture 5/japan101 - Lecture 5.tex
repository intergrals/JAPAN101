\documentclass{article}

\usepackage[utf8]{inputenc}
\usepackage{fullpage}
\usepackage{times}
\usepackage{tcolorbox}
\usepackage{enumitem}
\usepackage{bookmark}
\usepackage{CJKutf8}

\setlist{itemsep=2pt}
\renewcommand{\thesection}{Lecture \arabic{section}}
\renewcommand{\thesubsection}{\arabic{section}.\arabic{subsection}}
\setcounter{section}{4}

\begin{document}
 \begin{CJK}{UTF8}{min}

\noindent
{JAPAN 101 \hfill Hao Pan}\\
{Shimoda, Fumie}\\
{Spring 2018}

%%%%%%%%%%%%%%%%%%%%%%%%%%%%%%%%%%%

\begin{center}
\section{}
\noindent
{\hfill 30/05/2018 [W]}
\end{center}

\subsection{Things to do in Japan}

\begin{itemize}
\item すしをたぺます。
\item ふじさんをみます。
\item きものをかいます。
\end{itemize}

\subsection{Verb Groups}

\begin{enumerate}
\item U-verbs
\begin{itemize}
\item あう (to meet), ある (there is..), いく (to go), かえす* (to return), のむ (to drink), はなす (to speak)
\item In their conjugate form, drop the last character and add the one above it in the character table. Then add ます.
\begin{itemize}
\item o $\rightarrow$ e $\rightarrow$ u $\rightarrow$ i $\rightarrow$ a $\rightarrow$ o.
\end{itemize}
\item {\bf Ex.} ある $\rightarrow$ あります。
\end{itemize}
\item Ru-verbs
\begin{itemize}
\item Contains verbs ending with {\bf iru} and {\bf eru} (excluding かえる).
\item いる (to be in/stays at), おきる (to get up), たべる (to eat), ねる (to sleep)
\item In their conjugate form, drop the る and add ます.
\item {\bf Ex.} おきる $\rightarrow$ おきます。
\end{itemize}
\item irregular verbs
\begin{itemize}
\item くす (to come), する (to do), べんきょうする (to study)
\item These verbs have unique conjugations.
\item くす $\rightarrow$ きます。
\item する $\rightarrow$ します。
\item べんきょうする $\rightarrow$ べんきょうします。
\end{itemize}
\end{enumerate}

\subsection{Using Verbs in a Sentence}

\begin{itemize}
\item ([N] は) Obj. を v.masu-form ます。
\begin{itemize}
\item {\bf Ex.} {\it さかなをたべます。| I eat fish.}
\end{itemize}
\item Use ません for negative verb.
\begin{itemize}
\item {\bf Ex.} {\it さかなはたべません。| I do not eat fish.}
\end{itemize}
\end{itemize}

\clearpage

\underline{\bf Example Question}

\begin{itemize}
\item {\bf Q:} すぽうつをしますか。| Do you do (play) sports?
\item {\bf A:}
\begin{itemize}
\item はい、します。 | Yes, I do.
\item いいえ、しません。| No, I do not.
\end{itemize}
\end{itemize}

\subsection{Actions}

\begin{tabular}{ l | l }
Japanese & English\\
\hline
てにすをします。 & Playing tennis.\\
あさごはんをたべます。 & Eating breakfast.\\
こうひいをのみます。 & Drink coffee.\\
てれびをみます。 & Watch TV.\\
にほんごをべんぎょうします。 & Study Japanese.\\
しんぶんをよみます。 & Read newspaper.\\
CDをききます。& Listen to CD.\\
えいごをはなします。 & Speak English.\\
てがみをかきます。 & Write a letter.\\
くつをかいます。 & Buy shoes.\\
しゃしんをとります。 & Take a picture.\\
\end{tabular}

\subsection{Verb with Location}

\begin{itemize}
\item ([N] は) place で obj. を v.masu-form ます。
\begin{itemize}
\item {\bf Ex.} {\it としょかんでほんをよみます。| Read a book at the library.}
\end{itemize}
\item Locations:
\begin{itemize}
\item いえ/うち | Home
\item がっこう | School
\item きっさてん | Coffee shop
\end{itemize}
\end{itemize}

\subsection{Goal of Movement}

\begin{itemize}
\item Destination に/へ(e) [one of the following]
\begin{itemize}
\item いきます。 | to go.
\item きます。| to come.
\item かえります。 | to return.
\end{itemize}
\item {\bf Examples:}
\begin{itemize}
\item {\it としょかんへいきます。 | Go to the library.}
\item {\it がっこうへきます。 | Come to school.}
\item {\it きっさてんにきます。 | Come to coffee shop.}
\item {\it いえにかえります。 | Return home.}
\item {\it あめりかにかえります。 | Return to America.}
\end{itemize}
\end{itemize}

\end{CJK}
\end{document}