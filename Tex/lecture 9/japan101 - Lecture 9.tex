\documentclass{article}

\usepackage[utf8]{inputenc}
\usepackage{fullpage}
\usepackage{times}
\usepackage{tcolorbox}
\usepackage{enumitem}
\usepackage{multicol}
\usepackage{bookmark}
\usepackage{CJKutf8}
\usepackage[normalem]{ulem}

\bookmarksetup{
  numbered, 
  open,
}

\setlist{itemsep=2pt}
\renewcommand{\thesection}{Lecture \arabic{section}}
\renewcommand{\thesubsection}{\arabic{section}.\arabic{subsection}}
%\renewcommand{\thesubsubsection}{\indent\arabic{section}.\arabic{subsection}.\arabic{subsubsection}}
\setcounter{section}{8}

\begin{document}
 \begin{CJK}{UTF8}{min}

\noindent
{JAPAN 101 \hfill Hao Pan}\\
{Shimoda, Fumie}\\
{Spring 2018}

%%%%%%%%%%%%%%%%%%%%%%%%%%%%%%%%%%%

\begin{center}
\section{}
\noindent
{\hfill 04/07/2018 [W]}
\end{center}

\subsection{New Vocabulary}

\begin{tabular}{ | l | l | }
\hline
Japanese & English\\
\hline
たべもの & food\\
ホテル(ほてる) & hotel\\
レストラン(れすとらん) & Restaurant\\
うみ & sea\\
とうでしたか。 & How was it?\\
\hline
\end{tabular}

%%%%%%%%%%%%%%%%%%%%%%%%%%%%%%%%%%%
\subsection{Adjective Review}

\subsubsection{I-Adjectives}
\begin{itemize}
\item きょうは\uline{あつい}です。 | \emph{Today is \uline{hot}.}
\item きょうは\uline{さむくない}です。 | \emph{Today is \uline{not cold}.}
\end{itemize}

\subsubsection{NA-Adjectives}
\begin{itemize}
\item ひらがなは\uline{かんたん}です。 | \emph{Hiragana is \uline{simple}.}
\item ひらがなは\uline{かんたんじゃない}です。 | \emph{Hiragana is \uline{not simple}.}
\end{itemize}

%%%%%%%%%%%%%%%%%%%%%%%%%%%%%%%%%%%
\subsection{Past Tense}

\subsubsection{I-Adjectives}
[N]は[I-adj w/o い]
$\left\{
\begin{tabular}{@{} l @{}}
かたです。\\
くなかたです。
\end{tabular}
\right.$

\begin{itemize}
\item きのうは\uline{あつかたです}。 | \emph{Yesterday \uline{was hot}.}
\item テスト(てすと)は\uline{むずかしくなかた}です。 | \emph{The test \uline{was not difficult}.}
\end{itemize}

\subsubsection{NA-Adjectives}
[N]は[NA-adj]
$\left\{
\begin{tabular}{@{} l @{}}
でした。\\
じゃなかったです。
\end{tabular}
\right.$

\begin{itemize}
\item きのうは\uline{ひまでした}。 | \emph{Yesterday I \uline{was free}.}
\item きのうは\uline{ひまじゃなかったです}。 | \emph{Yesterday I \uline{was not free}.}
\end{itemize}

%%%%%%%%%%%%%%%%%%%%%%%%%%%%%%%%%%%
\subsection{Degree Adverbs}

[N]は[freq]
$\left\{
\begin{tabular}{@{} l @{}}
{[I-adj w/o い]}かたです。\\
{[NA-adj]}でした。
\end{tabular}
\right.$

\bigskip

\begin{tabular}{ | l | l || l | l | }
\hline
\multicolumn{4}{| c |}{Degree Adverbs}\\
\hline
たいへん & very (formal) & すこし & a little (formal)\\
とても & very (not formal) & ちょっと & a little (casual)\\
すごく & very (casual) & - & -\\
あまり* & not very & - & -\\
\hline
\end{tabular}
\begin{itemize}
\item[*] Use with negative sentence.
\end{itemize}

%%%%%%%%%%%%%%%%%%%%%%%%%%%%%%%%%%%
\subsection{Likes and Dislikes}

[Person]は[N]が
$\left\{
\begin{tabular}{@{} l l @{}}
すき & like\\
きらい & dislike\\
\end{tabular}
\right\}$
(じゃない)です。

\begin{itemize}
\item {\bf Q}: Pizza がすきですか。 | \emph{Do you like pizza?}
\end{itemize}
\begin{tabular}{ | l | l | }
\hline
はい、だいすきです。 & Yes, I love it.\\
はい、すきです。 & Yes, I like it.\\
いいえ、すきじゃないです。 & No, I do not like it.\\
いいえ、きらいです。 & No, I dislike it.\\
いいえ、だいきらいです。 & No, I hate it.\\
すきでもきらいでもないです。 & Neither like nor dislike.\\
\hline
\end{tabular}

\subsubsection{Favorites}

\begin{itemize}
\item すきな__は[N]です。 | \emph{My favorite \_\_ is \emph{[N]}.}
\end{itemize}

%%%%%%%%%%%%%%%%%%%%%%%%%%%%%%%%%%%
\subsection{Suggestions}

\begin{itemize}
\item {\bf Q}: コーヒーをのみましょうか。 | \emph{Shall we drink some coffee?}
\begin{itemize}
\item {\bf A}: いいですね。 | \emph{Yes.}
\item {\bf A}: そうしましょう。 | \emph{Yes.}
\item {\bf A}: すみませんが、ちょうと… | \emph{Sorry, I'm a little...}
\end{itemize}
\end{itemize}

%%%%%%%%%%%%%%%%%%%%%%%%%%%%%%%%%%%
\subsection{Counting}

\begin{itemize}
\item Flat and thin objects are counted using: [\#]まい
\item {\bf Ex.} きってを\uline{さんまい}ください。 | \emph{Please give me \uline{3 stamps}.}
\item {\bf Q}: なんまいありますか。 | \emph{How many are there?}
\begin{itemize}
\item {\bf A}: _まいあります。 | \emph{There are \_\_.}
\end{itemize}
\end{itemize}




\end{CJK}
\end{document}